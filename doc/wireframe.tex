\documentclass[12pt,a4paper]{article}
\usepackage[top=2cm, bottom=2cm, left=3.5cm, right=3.5cm]{geometry}
\usepackage[polish,english]{babel}
\usepackage[utf8]{inputenc}
\usepackage[T1]{fontenc}
\usepackage{polski}
\usepackage{amsfonts}
\usepackage{indentfirst}
\usepackage{amsmath}

\author{Kacper Gutowski}
\title{Projekt z~grafiki komputerowej\\*Kamera wirtualna: wire\dywiz frame}

\begin{document}%
\selectlanguage{polish}
\maketitle

\section{Wstęp}

Celem projektu było stworzenie kamery wirtualnej, czyli programu pozwalającego
oglądać obraz trójwymiarowej sceny. W~tej części projektu wymagana była jedynie
implementacja podstawowych przekształceń geometrycznych i~wyświetlenie
krawędzi obiektów (wire\dywiz frame) bez eliminacji elementów zasłoniętych.
Kamera miała mieć możliwość poruszania się i~obracania we wszystkich kierunkach
oraz zmiany ogniskowej (zoom).

\section{Wybór rozwiązań}

Program napisałem w~języku C89, ze względu na przenośność i~szybkość działania.
Do wyświetlania wykorzystałem bibliotekę XCB, która jest interfejsem dla protokołu X,
a~więc pozwala rysować linie i~poligony abstrahując od sprzętu i~jest uniwersalna
dla wszystkich systemów POSIX.

Opis sceny jest tablicą obiektów których definicje są wczytywane
z~zewnętrznych plików przy starcie programu.
W~pamięci obiekty przechowywane są jako chmura punktów oraz lista krawędzi pomiędzy nimi.
Właściwe przekształcenia zrealizowałem jako operacje liniowe na współrzędnych jednorodnych.

\section{Implementacja}

\subsection{Macierze}
Ponieważ wszystkie przekształcenia opierają się na operacjach macierzowych,
kluczowym z~punktu widzenia funkcjonalności programu jest moduł implementujący te operacje.
Potrzebne było: mnożenie macierzy $4\times4$ odpowiadające kompozycji przekształceń,
mnożenie macierzy przez wektor z~prawej strony we współrzędnych jednorodnych,
co odpowiada aplikacji przekształcenia, oraz normalizacja wektora po przekształceniu.
Ponieważ nic więcej nie było potrzebne, zrealizowałem te operacje samodzielnie
w~sposób pozwalający na ewentualną optymalizację.  Powstały dwie gałęzie kodu,
jedna w~,,czystym'' przenośnym C, druga zoptymalizowana dla jednostek SSE.

\subsection{GUI}
Interfejs użytkownika składa się z~pojedynczego okienka zawierającego
wyłącznie obraz sceny, bez żadnych dodatkowych kontrolek.
Krawędziowy obraz sceny jest rysowany białą kreską na czarnym tle
bez żadnych dodatkowych efektów.
Dla uzyskania płynnego odświeżania zastosowałem buforowanie
w~\mbox{PixMap} po stronie serwera X.\  Jest to rozwiązanie dużo prostsze
niż wykorzystanie rozszerzenia DBE, a~skuteczne.

Przy pomocy klawiatury lub myszy użytkownik może zmieniać położenie
i~orientację kamery.
Możliwe są następujące przesunięcia:
,,do~przodu'' i~,,do~tyłu'' wzdłuż kierunku patrzenia, osi prostopadłej do ekranu,
,,w~lewo'' i~,,w prawo'' wzdłuż osi równoległej do poziomej krawędzi okienka,
oraz ,,w górę'' i~,,w~dół'' wzdłuż osi równoległej do pionowej krawędzi okienka.
Dostępne obroty~to:
,,obrót w~lewo'' i~,,obrót w~prawo'' wokół osi równoległej do pionowej krawędzi okienka (\textit{yaw}),
,,obrót w~górę'' i~,,obrót w~dół'' wokół osi równoległej do poziomej krawędzi okienka (\textit{pitch}),
oraz
,,obrót zgodnie ze wskazówkami zegara'' i~,,przeciwnie do r.w.z.''\ wokół kierunku patrzenia
(\textit{roll}).

Ponieważ program nie miał być silnikiem gier, nie ma żadnych dodatkowych założeń o~wyglądzie świata.
Dlatego żaden z~kierunków poza układem kamery nie jest wyróżniony.
W~szczególności nie ma ustalonego kierunku do dołu (zwykle naturalnie wyróżniony przez grawitację).
Efektem tego założenia jest aktywna kompozycja obrotów (wszystkie obroty są wokół \textsl{aktualnej}
osi kamery), przez co sterowanie odbiega od tego znanego z~gier.

\subsubsection{Klawiatura}
\begin{itemize}
    \addtolength{\itemsep}{-0.5\baselineskip}
  \item zmiana położenia (przesunięcia)
    \begin{itemize}
      \item \texttt{W} lub $\uparrow$~--- do~przodu,
      \item \texttt{S} lub $\downarrow$~--- do~tyłu,
      \item \texttt{A} lub $\leftarrow$~--- w~lewo,
      \item \texttt{D} lub $\rightarrow$~--- w~prawo,
      \item \texttt{T}~--- w~górę,
      \item \texttt{G}~--- w~dół;
    \end{itemize}
  \item zmiana orientacji (obroty)
    \begin{itemize}
      \item \texttt{H}~--- obrót w~lewo,
      \item \texttt{J}~--- obrót w~dół,
      \item \texttt{K}~--- obrót w~górę,
      \item \texttt{L}~--- obrót w~prawo,
      \item \texttt{<}~--- obrót zgodnie z~r.w.z.,
      \item \texttt{>}~--- obrót przeciwny do r.w.z.;
    \end{itemize}
  \item zmiana długości ogniskowej (zoom)
    \begin{itemize}
      \item \texttt{+}~--- przybliżenie (zmniejszenie ogniskowej),
      \item \texttt{-}~--- oddalenie (zwiększenie ogniskowej);
    \end{itemize}
  \item inne
    \begin{itemize}
      \item \texttt{ESC}~--- zakończenie programu.
    \end{itemize}
\end{itemize}

\subsubsection{Mysz}
Zmiana położenia możliwa jest poprzez ruch myszy przy wciśniętym
prawym przycisku. Ruchy myszy odpowiadają przesunięciom w~płaszczyźnie równoległej
do kierunku patrzenia i~poziomej krawędzi ekranu.

Obroty kamery wykonywane są podczas ruchów myszą z~lewym przyciskiem wciśniętym.
Ruch w~lewo\dywiz prawo odpowiada obrotom wokół pionowej osi kamery (\textit{yaw}),
natomiast do~przodu\dywiz  do~tyłu odpowiada obrotom wokół poziomej (\textit{pitch}).
Ruch w~innym kierunku powoduje odpowiednie złożenie tych obrotów.
Jest to istotnie inne sterowanie niż typowe dla gier, gdzie ruch myszy odpowiada
przesunięciu końca wektora wzroku po sferze $(\varphi,\vartheta)$.

Ponadto możliwa jest zmiana ogniskowej poprzez ruch z~środkowym przyciskiem wciśniętym.

\subsection{Kamera}
Transformacja obiektów umieszczonych w~przestrzeni do układu ekranu wykonywana
jest przy pomocy operacji liniowej zaaplikowanej do każdego wierzchołka.

\subsubsection{Stan kamery}
Ta pojedyncza operacja wyznaczana jest jako złożenie następujących:
\begin{itemize}

  \item translacja początku układu współrzędnych do położenia kamery
    \[ A = \left( \begin{array}{cccc}
      1 & 0 & 0 & -c_x \\*
      0 & 1 & 0 & -c_y \\*
      0 & 0 & 1 & -c_z \\*
      0 & 0 & 0 & 1
    \end{array} \right)
    \;\;\mbox{gdzie}\;\;
    c = \left( \begin{array}{c}
      c_x \\ c_y \\ c_z
    \end{array} \right)
    \]
    jest położeniem kamery we współrzędnych świata.

  \item orientacja kamery
    \[ B = \left( \begin{array}{cccc}
      x_x & x_y & x_z & 0 \\*
      y_x & y_y & y_z & 0 \\*
      z_x & z_y & z_z & 0 \\*
      0 & 0 & 0 & 1
    \end{array} \right) 
    \;\;\mbox{gdzie}\;\;
    \overset{\rightarrow}{x} = \left( \begin{array}{c}
      x_x \\ x_y \\ x_z
    \end{array} \right),\;
    \overset{\rightarrow}{y} = \left( \begin{array}{c}
      y_x \\ y_y \\ y_z
    \end{array} \right),\;
    \overset{\rightarrow}{z} = \left( \begin{array}{c}
      z_x \\ z_y \\ z_z
    \end{array} \right),
    \]
    to wersory osi kamery, odpowiednio
    $\overset{\rightarrow}{x}$ kierunek w~prawo,
    $\overset{\rightarrow}{y}$ kierunek w~dół,
    $\overset{\rightarrow}{x}$ kierunek patrzenia.\newline
    $ |\overset{\rightarrow}{x}|=|\overset{\rightarrow}{y}|=|\overset{\rightarrow}{z}|=1 $.
    $\;\; \overset{\rightarrow}{x}\cdot\overset{\rightarrow}{y} = 
    \overset{\rightarrow}{y}\cdot\overset{\rightarrow}{z} = 
    \overset{\rightarrow}{z}\cdot\overset{\rightarrow}{x} = 0$.
    
  \item rzut środkowy na płaszczyznę prostopadłą do kierunku patrzenia $\overset{\rightarrow}{z}$
      \[ C = \left( \begin{array}{cccc}
        1 & 0 & 0 & 0 \\*
        0 & 1 & 0 & 0 \\*
        0 & 0 & 1 & 0 \\*
        0 & 0 & \frac{1}{f} & 0 \\*
      \end{array} \right)
      \]
      gdzie $f$ jest długością ogniskowej. Odpowiada odległości rzutni od obserwatora.

  \item dostosowanie do współrzędnych ekranowych
    \[ C = \left( \begin{array}{cccc}
      s_x & 0 & 0 & \frac{W}{2} \\*
      0 & s_y & 0 & \frac{H}{2} \\*
      0 & 0 & 1 & 0 \\*
      0 & 0 & 0 & 1 \\*
    \end{array} \right) \]
    gdzie $W$ i~$H$ to odpowiednio szerokość i~wysokość okienka w~którym obraz ma być wyświetlony,
    a~$s_x$ i~$s_y$ to skala w~pikselach na jednostkę.
\end{itemize}

Każdy punkt który ma zostać narysowany podlega transformacji macierzą
\[ P = QR = (DC)(BA) \]
Macierz ta musi zostać na nowo wyznaczona za każdym razem gdy zmieni się któryś z~parametrów,
dlatego obliczenia zostały zorganizowane w~drzewo binarne z~bitem ważności
zgodnie z~kolejnością oznaczoną powyżej ($Q=DC$, $R=BA$) i~tylko nieaktualne węzły
są przeliczane na nowo.
W~przypadku czterech macierzy nie przyspiesza to bardzo obliczeń,
ale pozwala łatwo dodać nowe przekształcenia w~dowolnym miejscu bez konieczności
przebudowy kodu.

\subsubsection{Modyfikacja parametrów kamery}

Modyfikacja długości ogniskowej wykonywana jest bezpośrednio:
\mbox{$f_{t+1} = f_t + \Delta$} gdzie $\Delta$ to współczynnik ustalony przez GUI
zależny od kierunku i~wielkości zmiany.

Tak samo przesunięcia wzdłuż wersorów układu kamery wyznaczane są jako:
$c_{t+1} = c_t + \overset{\rightarrow}{u}\cdot\Delta $
gdzie $\overset{\rightarrow}{u}$ to, zależnie od wybranego kierunku,
$\overset{\rightarrow}{x}$, $\overset{\rightarrow}{y}$ lub~$\overset{\rightarrow}{z}$.

Obroty wokół osi kamery realizowane są przez:
\[ B_{t+1} = \left(R(\overset{\rightarrow}{u},\varphi) {B_t}^T\right)^T
= B_t {R(\overset{\rightarrow}{u},\varphi)}^T \]
gdzie ${R(\overset{\rightarrow}{u},\varphi)}^T$ to transponowana macierz obrotu
wokół osi $\overset{\rightarrow}{u}$ o~kąt $\varphi$.
Konieczność transponowania wynika z~tego, że wersory
$\overset{\rightarrow}{x}\overset{\rightarrow}{y}\overset{\rightarrow}{z}$
znajdują się w~wierszach macierzy $B$, a~nie w~kolumnach.
Żeby uniknąć tej dodatkowej pracy, korzystam z~wyżej przedstawionej zależności
i~generuję od razu transponowaną macierz obrotu, którą mnożę prawostronnie.

\subsection{Model}
Modele obiektów są wczytywane z~plików.
Użyłem prostego tekstowego formatu, w~którym
po kolei zdefiniowane są wierzchołki,
krawędzie (pary wierzchołków) i~powierzchnie (listy krawędzi).
Taki format pozwala utworzyć dowolną reprezentację po wczytaniu.

Aktualnie używana reprezentacja w~pamięci to prosta tablica krawędzi
zawierająca pary numerów wierzchołków, oraz tablica wierzchołków,
która zawiera ich współrzędne.
Informacja o~powierzchniach jest na razie ignorowana, gdyż nie jest
potrzebna do narysowania wire-frame.

Przy rysowaniu wire\dywiz frame najpierw wszystkie wierzchołki są transformowane
do układu ekranu, a~następnie lista krawędzi jest przeglądana po kolei
i~krawędzie widoczne na ekranie są rysowane.
Widoczność jest wyznaczana już po rzutowaniu przy użyciu dwuwymiarowego
algorytmu Cohena\dywiz Sutherlanda z~modyfikacją pozwalającą łatwo ominąć
krawędzie znajdujące się z~tyłu kamery.

\subsection{Sfera}
W~ramach projektu napisałem też prosty skrypt w~perlu,
generujący aproksymację sfery dowolnej dokładności,
trójkątami w~przybliżeniu równej wielkości.
Skrypt zaczyna z~czworościanu foremnego wpisanego w~sferę
a~następnie przez sukcesywny podział ścian na mniejsze trójkąty,
w~kolejnych iteracjach zbliża się do sfery.
Wynik działania skryptu formatowany jest w~postaci czytelnej
dla głównego programu.

Przykład takiej sfery po czterech iteracjach (powierzchnia z~514 trójkątów)
jest widoczny na domyślnej scenie programu.

\section{Podsumowanie i~wnioski}

Program działa poprawnie i~zadowalająco szybko.

\end{document}
% vim: set spell spl=pl,en ft=tex :
