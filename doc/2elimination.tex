% vim: set ft=tex spell spl=pl,en :
\documentclass[12pt,a4paper]{article}
\usepackage[top=2cm, bottom=2cm, left=3.5cm, right=3.5cm]{geometry}
\usepackage[polish,english]{babel}
\usepackage[utf8]{inputenc}
\usepackage[T1]{fontenc}
\usepackage{polski}
\usepackage{amsfonts}
\usepackage{indentfirst}
\usepackage{amsmath}

\author{Kacper Gutowski}
\title{Projekt z~grafiki komputerowej\\*
Kamera wirtualna: \\ eliminacja powierzchni zasłoniętych}

\begin{document}%
\selectlanguage{polish}
\maketitle

\section{Wstęp}
Celem tej części projektu była implementacja
algorytmu zasłaniania powierzchni zasłoniętych
aby rysować obiekty z~wypełnionymi ścianami,
a~nie tylko ich wire\dywiz frame.

\section{Wybór rozwiązań}
Program oparty jest na poprzedniej części projektu
i~stanowi jego naturalną rozbudowę, dlatego zastosowałem
te same rozwiązania.
Program napisany jest w~języku C89, przeznaczony jest dla systemów
zgodnych z~POSIX i~wykorzystuje bibliotekę XCB do komunikacji z~serwerem~X.

Zasłanianie zrealizowałem przy pomocy drzewa BSP.
Algorytm ten pozwala w~nieskomplikowany sposób
poprawnie ustalić kolejność rysowania dla dowolnego
układu obiektów, również z~zasłanianiem cyklicznym.

\section{Implementacja}
\subsection{Model}
Tak jak w~poprzedniej części projektu, scena zdefiniowana jest
statycznie w~kodzie.  Jej opis składa się z~trzech tablic
zawierających odpowiednio, nazwę pliku z~modelem,
macierz przekształcającą model do postaci w~jakiej ma funkcjonować
na scenie, oraz kolor obiektu.

W~stosunku do poprzedniej części projektu zmieniłem
format plików z~modelami, gdyż okazał się jednak nie być wygodnym.
Aktualnie wczytywane mogą być pliki w~popularnym formacie MESH.

W~pamięci każdy model przechowywany jest jako lista
wierzchołków oraz lista powierzchni, przy czym tym razem
powierzchnie mogą być wyłącznie trójkątne
(sekcja \texttt{Quadrilaterals} pliku MESH jest automatycznie dzielona na trójkąty).

\subsection{Drzewo BSP}
Ze wczytanych modeli tworzona jest scena w~postaci listy wszystkich wierzchołków
oraz drzewa BSP zawierającego powierzchnie wszystkich obiektów.

Drzewo BSP, w~każdym wierzchołku, dzieli aktualną przestrzeń
na lewą i~prawą podprzestrzeń, które oddzielone są płaszczyzną wyznaczoną
przez jedną z~powierzchni. Dzięki temu przeglądając drzewo we właściwej
kolejności, możliwe jest rysowanie kolejnych powierzchni zakładając
że wielokąty narysowane później na ekranie zamazują poprzednie.

\subsubsection{Podział powierzchni}
Jeśli jedna ze ścian na scenie przecina aktualną powierzchnię
dzielącą przestrzeń, nie może zostać zakwalifikowana do prawego
lub lewego poddrzewa.  Taka ściana musi zostać podzielona
na dwie części wzdłuż płaszczyzny dzielącej.

Dowolny trójkąt przecięty płaszczyzną może zostać podzielony na
dwa trójkąty lub trójkąt i~czworokąt wypukły.
Prosty algorytm pozwala wyznaczyć punkty przecięcia krawędzi
dzielonego trójkąta z~powierzchnią dzielącą, będą one nowymi
wierzchołkami.
Jeśli produktem podziału jest czworokąt to dzielony jest
na dwa trójkąty tą są procedurą, która stosowana jest przy
wczytywaniu modeli.

Dowolny czworokąt wypukły można podzielić na dwa trójkąty
na dwa sposoby, dzieląc wzdłuż jednej lub drugiej przekątnej.
Wybierany jest zawsze podział wzdłuż krótszej przekątnej,
gdyż pozwala to uniknąć zdegenerowanych trójkątów.

Największą trudnością przy implementacji drzewa BSP
jest właściwe zarządzanie pamięcią przy dzieleniu trójkątów,
gdzie pojawiają się nowe wierzchołki i~powierzchnie.
W~tym zadaniu widać przewagę języków z~\textsl{garbage collectorem}.

\subsubsection{Wybór wierzchołka}
Zbudowanie optymalnego drzewa BSP jest problemem klasy~NP.
Kluczowym dla jego rozwiązania jest wybór optymalnej
płaszczyzny dzielącej.

Są dwa sprzeczne kryteria które należy wziąć pod uwagę:
zbalansowanie drzewa, oraz liczba wielokątów, które przecinają
płaszczyznę dzielącą i~które trzeba będzie podzielić.

Do wyboru wierzchołka na każdym poziomie, spośród dostępnych
płaszczyzn wybierana jest próbka 20 kandydatów i~liczona
jest liczba powierzchni która znajdzie się w~lewym
i~prawym poddrzewie oraz tych przeciętych.

Spośród kandydatów wybierana jest ostatecznie ta płaszczyzna,
która minimalizuje
$ \left| n_{left} - n_{right} \right| + k n_{split} $
gdzie $n_{left}$, $n_{right}$ i~$n_{split}$ to wymienione wcześniej
liczby powierzchni,
a~$k$ jest magicznym współczynnikiem, ustawionym na~$2$. % lol :)

\subsection{Kolory}
Ponieważ rysowane są wypełnione ściany bez wyróżniania krawędzi,
konieczne dla czytelności wygenerowanego obrazu było zróżnicowanie
ich kolorów.
Zwykle realizowane jest to przez jakiś model oświetlania,
jednak tutaj zastosowałem proste rozwiązanie imitujące
oświetlenie którego źródłem jest cała płaszczyzna równoległa
do rzutni znajdująca się za plecami obserwatora.

Każdy obiekt ma częściowo ustalony stały kolor we współrzędnych $u^*_0v^*_0$.
Dla danego rzutu ustalony kolor wynikowy zależy od kwadratu odległości
środka ciężkości powierzchni od rzutni oraz od rzutu wektora normalnego.
Wyznaczony kolor we współrzędnych CIE~$L^*u^*v^*$ jest następnie
przekształcany na {s$RGB$} i~używany do rysowania ściany.

Wybrałem przestrzeń CIE~$L^*u^*v^*$, gdyż tak opisany kolor zarówno łatwo
przekształcić do $RGB$, jak i~łatwo w~niej uzyskać pożądane efekty, czyli
zmienić jasność (skalowanie $L^*$)
oraz saturację (równomierne skalowanie $u^*$ i~$v^*$).

\subsection{Scena}
Scena wygląda podobnie jak w~poprzedniej części projektu.
Zawiera cztery sześciany oraz aproksymację sfery
(tym razem z~czworokątnymi bokami, na bazie sześcianu).
Żeby pokazać poprawność implementacji BSP, 
dodatkowo wstawiłem klasyczny przykład zasłaniania cyklicznego
z~czterech trójkątów.

\section{Podsumowanie}
Program działa szybko i~poprawnie.
% i chuj

\end{document}
